\PassOptionsToPackage{table}{xcolor}
\documentclass[10pt,a4paper,landscape]{article}
\usepackage{multicol}
\usepackage{calc}
\usepackage{ifthen}
\usepackage[landscape]{geometry}
\usepackage{titlesec}
\usepackage{graphicx}
\usepackage{ stmaryrd }
\usepackage{tikz}
\usepackage[T1]{fontenc}
\usetikzlibrary{shadows, calc}

%\usepackage[table]{xcolor}

% Bitbake cheat sheet

% This sheet is meant to be compiled with pdflatex.

% This code is based on the LaTeX cheat sheet found here :
% http://www.stdout.org/~winston/latex/
% by Winston Chang

% Licensed under Creative Commons Attribution-NonCommercial-ShareAlike 3.0 Unported License.

% Setting the margins at 1cm everywhere
\geometry{top=1cm,left=1cm,right=1cm,bottom=1cm} 

% Turn off header and footer
\pagestyle{empty}
 
\makeatletter

\newcommand{\vcenteredinclude}[2]{\begingroup
\setbox0=\hbox{\includegraphics[#2]{#1}}%
\parbox{\wd0}{\box0}\endgroup}

\newcommand*\keystroke[1]{%
  \tikz[baseline=(key.base)]
    \node[%
      draw,
      fill=white,
      drop shadow={shadow xshift=0.25ex,shadow yshift=-0.25ex,fill=black,opacity=0.75},
      rectangle,
      rounded corners=2pt,
      inner sep=1pt,
      line width=0.5pt,
      font=\scriptsize\sffamily
    ](key) {~#1~\strut}
  ;
}

% Don't print section numbers
\setcounter{secnumdepth}{0}


\setlength{\parindent}{0pt}
\setlength{\parskip}{0pt plus 0.5ex}

\renewcommand{\familydefault}{\sfdefault}

\titleformat{\section}[block]%              
    {\tikz[overlay] \shade[left color=gray!10,right color=gray] (0,-1ex) rectangle (\linewidth,1.5em); \Large}%    
    {\thesection}%                   
    {1em}%
    {}

\titleformat{\subsection}[block]%              
    {\large}%    
    {\thesubsection}%                   
    {1em}%
    {}
% -----------------------------------------------------------------------

\begin{document}

%\renewcommand{\familydefault}{\sfdefault}

\newcommand{\ret}{\keystroke{$\hookleftarrow$}}
\newcommand{\shift}{\keystroke{$\Uparrow~$}}
\newcommand{\alt}{\keystroke{Alt}}
\newcommand{\up}{\keystroke{$\uparrow$}}
\newcommand{\down}{\keystroke{$\downarrow$}}
\newcommand{\bkspc}{\keystroke{$\longmapsfrom$}}
\newcommand{\ctrl}[1]{\texttt{\keystroke{Ctrl}#1}}

\raggedright
\footnotesize
\begin{multicols}{2}

\begin{center}
     \Large{\textbf{Bitbake Cheat Sheet}} \\
\end{center}

\rowcolors{1}{gray!20}{white}

\section{Command Line options}

\begin{tabular}{p{5cm}p{\linewidth - 5.9cm}}
-c \textless task \textgreater & execute \textless task \textgreater for the image or recipe being built. ex: bitbake -c fetch busybox. Some of the possible tasks are: fetch, configure, compile, package, clean \\
-f & force execution of the operation, even if not required \\
-v & show verbose output \\
-DDD & show lots of debug information \\
-s & show recipe version information \\
--help & get usage help \\
-c listtasks \textless image-or-recipe-name \textgreater & show the tasks associated with an image or individual recipe \\
-g \textless recipe \textgreater & output dependency tree in graphviz format \\
\end{tabular}

\section{Usefull commands}
\begin{tabular}{p{5cm}p{\linewidth - 5.9cm}}
bitbake foo & this builds package foo \\
bitbake -c clean foo & This command will clean up your tmp dir for the given package. It is very useful if you work on a new .bb recipe. Without it your changes to the recipe may not work. \\
bitbake -f -c compile foo & To recompile your source code if you change a line in it. \\
bitbake -s | grep foo & Check the version of the recipe \\
bitbake -c devshell \textless target \textgreater &  opens the interactive devshell in the source directory of target, this is usefull for debugging build failures \\
bitbake virtual/kernel -c menuconfig & Interactive kernel configuration \\
bitbake \textless image \textgreater -c fetchall & Fetch all sources for that particular image. This should be execute before a long running build.


\end{tabular}

\section{.bb file syntax}
\begin{tabular}{p{5.5cm}p{\linewidth - 6.4cm}}
VAR = ``foo'' & simple assignment \\
VAR ?= ``foo'' & assign if no other value is already assigned (default assignment) \\
VAR ??=``foo'' & weak default assignment, takes lower precedence than ?= \\
VAR = ``stuff \$\{OTHER\_VAR\} more'' &  variable expansion, OTHER\_VAR expanded at time of reference to VAR \\
VAR := ``stuff \$\{OTHER\_VAR\} more'' & immediate variable expansion, OTHER\_VAR expanded at time of parsing this line  \\
VAR += ``foo'' & append with space \\
VAR =+ ``foo'' & prepend with space \\
VAR .= ``foo'' & append without space \\
VAR =. ``foo'' & prepend without space \\
VAR\_append = ``foo'' & append without space \\
VAR = ``foo \$\{@\textless line-of-python-code \textgreater\}'' & python code expansion, ex: VAR = ``the date is: \$\{@time.strftime(`\%Y\%m\%d',time.gmtime())\}'' \\
include foo & include file, include file named ``foo'', search BBPATH \\
require [\textless path\textgreater]foo & require file, include file named ``foo'', failing if not found exactly where specified \\
inherit foo & inherit classes, include definitions from foo.bbclass \\
do\_sometask() \{   \textless shell code\textgreater  \}  & define a task using shell code \\
python do\_sometask \{ \textless python code\textgreater  \}  & define a task using python code \\
addtask sometask (before|after) other\_task & add a task, adds a defined task to the list of tasks, with the ordering specified. Zero or more ``before'' or ``after'' clauses can be used. \\
VAR[some\_flag]=``foo'' & associate a subsidiary flag value to a variable, a few subsidiary flag value names are well-defined: ``dirs'', ``cleandirs'', ``noexec'', ``nostamp'', ``fakeroot'', ``umask'', ``deptask'', ``rdeptask'', ``recdeptask'', ``recrdeptask''
Flag values appear to be used exclusively with task definitions (i.e. do\_sometask)
\end{tabular}



\rule{\linewidth}{0.25pt}
\scriptsize

Copyright \copyright\ 2014 Christian Ege

%http://www.gsmproductions.org/misc/sublime.html

\end{multicols}
\end{document}
